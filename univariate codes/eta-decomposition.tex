

\begin{theorem}[polynomial eta-decomposition]
    Any polynomial $p$ may be decomposed by any non-constant polynomial $\Eta$.
    Letting $\eta = \deg(\Eta)$ we may decompose $p$ into $\eta$ polynomials $p_0,\dots,\p_{\eta-1}$ all of degree less than $\deg(p)/\eta$ such that
    \begin{align}
        p(x) = \sum_{i=0}^{\eta-1} p_i(\Eta(x))x^i
    \end{align}
    \begin{remark}
        The symbol `$\Eta$' is not the uppercase Latin letter `H' but rather the uppercase Greek letter `Eta,' while `$\eta$' is the corresponding lowercase letter.
        We choose $\eta$ due to its analogous role in the polynomial decomposition of \cite{BBHR18}.
    \end{remark}

    \proof
    First invoke the following recursive algorithm on $p$ to obtain a decomposition of the form
    \begin{align}
        p(x) = \sum_{i=0}^{\floor{\deg(p)/\eta}} p'_i(x)\Eta(x)^i
    \end{align}
    where $p'_0,\dots,p'_{\floor{\deg(p)/\eta}}$ are all of degree less than $\eta$.

    \begin{itemize}
        \textbf{Decompose}.
        \\
        Input: polynomial $t$.
        \\
        Output: polynomials $t_0,\dots,t_{\floor{\deg(t)/\eta}}$ with $\deg(t_i)<\eta$ for all $i$ such that
        \begin{align}
            t(x) = \sum_{i=0}^{\floor{\deg(t)/\eta}} t_i(x)\Eta(x)^i
        \end{align}
        
        \item
        Divide $t$ by $\Eta$ to get quotient $q$ and remainder $r$ such that $\deg(q)\leq\deg(t)$ and $\deg(r)<\eta$.
        Note $\deg(q)+\eta = \deg(t)$.

        \item
        Invoke \textbf{Decompose} on $q$ to get $q_0,\dots,q_{\floor{\deg(q)/\eta}}$ with $\deg(q_i)<\eta$ for all $i$.

        \item
        For any $\alpha\in\R$ we have the identity $\floor{\alpha}+1 = \floor{\alpha+1}$, yielding
        \begin{aling}
            \floor{\deg(q)/\eta}+1 = \floor{\deg(q)/\eta+1} = \floor{\deg(t)/\eta
        \end{align}.
        Now $t$ may be expressed as
        \begin{align}
            t(x)
            &= r(x) + \Eta(x)\sum_{i=0}^{\floor{\deg(q)/\eta}} q_i(x)\Eta(x)^i \\
            &= r(x) + \sum_{i=1}^{\floor{\deg(q)/\eta}+1} q_{i-1}(x)\Eta(x)^i \\
            &= r(x) + \sum_{i=1}^{\floor{\deg(t)/\eta}} q_{i-1}(x)\Eta(x)^i
        \end{align}
        Return $t_0,\dots,t_{\floor{\deg(t)/\eta}} = r,q_0,\dots,q_{\floor{\deg(t)/\eta}}$, noting they satisfy the promised properties.
    \end{itemize}

    The current decomposition is a polynomial with respect to $\Eta(x)$ of degree less than $\deg(t)/\eta$ with coefficients as polynomials with respect to $x$ of degree less than $\eta$. 
    \\
    The desired decomposition is a polynomial with respect to $x$ of degree less than $\eta$ with coefficients as polynomials with respect to $\Eta(x)$ of degree less than $\det(t)/\eta$.
    \\
    Thus the current decomposition may be transformed into the desired decomposition by reinterpretation.
    Distribute powers of $\Eta(x)$ and then factor out powers of $x$. 
    Let $p_0,\dots,p_{\eta-1}$ be the coefficient polynomials of the result.
\end{theorem}


\begin{theorem}[eta-decomposition by interpolation]
    If the preimage of $\alpha$ under $\Eta$ has size at least $\eta$ then interpolating $p$ over this preimage yields a polynomial with coefficients $p_0(\alpha),\dots,p_{\eta-1}(\alpha)$, namely the $\eta$-decomposition polynomials of $p$ on $\alpha$.

    \proof
    Let $Y = \Eta^{-1}(\alpha)$, and suppose $|Y|\geq\eta$.
    Then $p$ restricted to $Y$ yields the polynomial
    \begin{align}
        p(y) = \sum_{i=0}^{\eta-1} p_i(\alpha)y^i
    \end{align}
    This polynomial of degree less than $\eta$ may be interpolated over any $\eta$ values of $Y$, thus extracting coefficients $p_0(\alpha),\dots,p_{\eta-1}(\alpha)$.
\end{theorem}








